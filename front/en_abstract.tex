\begin{abstractEn}
	In recent years, Permanent Magnet Synchronous Machine (PMSM) has become an attractive candidate for various industrial applications due to its high efficiency and torque density. 
	
	There are normally two kinds of control methods for PMSMs: Scalar Control methods and Vector Control methods. Scalar Control methods, such as V/F and I/F are easy to implement, but suffer from poor performance. For high performance PMSM drive systems, a Vector Control method known as Field Oriented Control (FOC) is often chosen as the control method and it offers high control dynamics. 
	
	This thesis focuses on the FOC of PMSM. First, the mathematical model of a PMSM in phase abc reference frame is introduced, then reference frame theory is introduced and applied to transform the mathematical model in abc frame to the stationary $\alpha\beta$ reference frame and the rotating dq reference frame. Based on the dq reference frame model of a PMSM, the FOC control principle is studied, and the controller design method for inner current loop is studied and analyzed in detail. Also, a back-EMF based rotor position estimation method is analyzed in detail.

	Then, a Matlab/Simulink model of the control system is built and simulation is run to verify the FOC principle, the controller design  
	and also the position estimation method.
	
	Finally, a dSPACE based control system is built and experiments are carried out to realize the FOC of PMSM, which lay a good fundation for further study of advanced control method of PMSM.
	
	\begin{flushright}
		Chen Dunzhi(Power Electronics and Drives)
		
		Directed by Prof. Weimin Wu
    \par\end{flushright}
\end{abstractEn}
\keywordsEn{dSPACE; Permanent Magnet Synchronous Machine; Field Oriented Control; Back-EMF based rotor position estimation}
