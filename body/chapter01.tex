\chapter{绪论}\label{ch:intr}
本章首先分析了选题背景与意义,然后介绍了矢量控制的发展概述。最后,确定了本文的主要研究内容。
\section{课题研究背景与意义}
直流电机具有良好的调速性能,但其结构复杂,需要定期维护,成本高。随着电力电子技术的发展,交流调速系统在众多场合已经替代直流电机。在众多的交流电机类型中,永磁同步电机利用永磁体励磁,与交流异步电机和传统绕线型同步电机相比,具有结构简单、体积小、质量轻、功率因数高、功率密度高等特点,在新能源汽车、工业机器人等各种应用场合受到广泛关注\upcite{dai_development_2007,msaddek_design_2015,nerg2014direct}。

电机控制技术是电机应用的重要环节,对于永磁同步电机,矢量控制是一种应用最为普遍,控制性能优越的控制方式,目前绝大多高性能永磁同步电机驱动器都采用矢量控制方法。随着工业4.0智能制造、电动机车、机器人等行业的飞速发展,未来将会对电机控制提出更高的控制需求。因此研究永磁同步电机矢量控制技术兼具研究与实用价值。

在此背景下,本文主要研究永磁同步电机矢量控制技术,通过应用坐标变换理论建立永磁同步电机dq坐标下的数学模型,根据数学模型分析矢量控制方法,并建立基于Simulink的永磁同步电机矢量控制模型进行仿真分析仿真,搭建实际实验平台,设计控制器,对理论仿真分析进行实验验证,为后续研究高性能永磁电机矢量控制技术打下坚实基础。
\section{矢量控制发展概述}
他励直流电机,其励磁与转矩没有耦合,独立控制,采用双闭环的方法具有良好调速性能\upcite{hughes2013electric}。交流电机具有非线性、强耦合的特点,控制十分复杂。坐标变换理论为电机控制带来了变革。利用坐标变换,可以将三相电机等效成两相电机,将励磁和转矩解耦,三相交流电机可以等效成为直流电机控制,这种解耦控制方法,即所谓的矢量控制\upcite{book1,vas1990vector,boldea2008active}。

在磁场定向矢量控制(FOC)中,需要反馈转子位置信息用来执行坐标变换,同时在转速闭环控制中,需要转速反馈。通常这些信息通过处理编码器信号得到,但是编码器安装不便、成本高等原因使得无位置传感器矢量控制成为当前发展的热点\upcite{review}。

所谓无位置传感器矢量控制是指不用编码器,转子位置、转速信息通过电压指令和采样电流计算得到。通常存在基于反电动势位置估算法和基于高频电压注入位置估算法两种\upcite{xie2015design}。

其中,基于反电动势的方法根据$\alpha\beta$电压电流信息计算转子位置\upcite{genduso_back_2010},通常再经过PLL来估算转速\upcite{chen_design_2010}。研究学者们提出了各种新型基于反电动势位置估算方法,主要包括、模型预测控制、扩展卡尔曼滤波、观测器、神经网络等方法\upcite{1,2,3,4}。

反电动势位置估算方法基于电机模型,可以采取多种不同的方法来降低估算误差,但是在低转速时,反电动势中噪声占主导成分,此时该方法不再适用。因此低转速时,采用基于高频电压注入法,高频电压注向电机注入高频电压信号,读取响应电流,利用信号处理的手段分离出电流响应中转子位置信息\upcite{ni_square-wave_2017,xie_minimum-voltage_2016,xie2015improved}。
\section{本文主要工作}
本课题研究设计永磁同步电机矢量控制系统。在理解永磁同步电机结构,深入理解其数学模型的基础之上,运用励磁电流和转矩电流解耦控制的矢量控制基本思想设计永磁同步电机双闭环矢量控制调速系统,搭建Simulink模型运行仿真验证其功能并搭建实际实验平台进行实验验证,为后续研究高性能永磁同步电机驱动技术打下基础。
本文的主要工作内容有:
\begin{enumerate}[label={(\arabic*)}]
	\item 介绍了永磁同步电机abc坐标数学模型,坐标变换理,应用坐标变换理论,导出不同坐标下的永磁同步电机数学模型。
	\item 运用永磁同步电机动态数学模型,建立了永磁同步电机矢量控制仿真模型,分析了电流环,速度环控制器设计方法。
	\item 详细分析了基于反电动势永磁同步电机转子位置估算方法,分析积分漂移补偿方法。
	\item 搭建基于dSPACE实时仿真系统的永磁同步电机矢量控制平台,实现电流转速双闭环矢量控制、实现基于反电动势位置估算的永磁电机矢量控制。
\end{enumerate}