\chapter{总结与展望}
\section{工作总结}
\begin{itemize}

\item 本文详细介绍了永磁同步电机数学建模,介绍了$\alpha\beta$坐标系dq坐标系的永磁同步电机数学模型,详细导出了定子电压方程、磁链方程和电磁转矩方程。
\item 根据dq坐标下的永磁同步电机数学模型,分析了永磁同步电机矢量控制基本原理,着重介绍了电流内环、转速外环PI参数设计原则,并详细分析了基于反电动势的永磁同步电机转子位置估算方法。
\item 根据永磁同步电机dq数学模型和矢量控制原理分析,搭建了基于simulink永磁同步电机矢量控制仿真模型,并验证了矢量控制原理、验证了基于反电动势位置估算算法。
\item 基于dSPACE/DS1103平台,搭建了永磁同步电机矢量控制平台,设计调节电流内环、转速外环PI参数实现双闭环转速控制,实现了基于反电动势转子位置估算,为后续开发高性能永磁同步电机驱动技术做了一定基础工作。
\end{itemize}
\section{工作展望}
由于时间有限,还有许多电机控制算法比如说基于高频电压注入法的转子位置估算方法等都没有来得及研究。而且由于永磁同步电机驱动系统比较复杂,本文分析的矢量控制建立在简化的模型基础之上,许多非线性因素,比如说逆变器死区非线性补偿,永磁同步电机电感参数随着铁芯饱和情况改变、数字控制带来的延时对控制性能的影响等等都没有考虑。这些因素都可能对实际电机运行效果带来不良影响,需要进一步研究探讨。





